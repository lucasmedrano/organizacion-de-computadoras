\documentclass[a4paper, 12pt]{article}

\usepackage[utf8]{inputenc}
\usepackage[spanish]{babel}
\usepackage[margin=2cm]{geometry}
\usepackage{graphicx}
\usepackage{float}
\usepackage{pdfpages}
\usepackage{listings}
\usepackage{listingsutf8}
\usepackage{xcolor}

\definecolor{mGreen}{rgb}{0,0.6,0}
\definecolor{mGray}{rgb}{0.5,0.5,0.5}
\definecolor{mPurple}{rgb}{0.58,0,0.82}
\definecolor{backgroundColour}{rgb}{0.95,0.95,0.92}

\lstset{
language=C,
%backgroundcolor=\color{backgroundColour},   
commentstyle=\color{mGreen},
%keywordstyle=\color{magenta},
keywordstyle=\color{blue},
%numberstyle=\tiny\color{mGray},
stringstyle=\color{mPurple},
tabsize=4,
basicstyle=\fontsize{11}{13}\ttfamily\footnotesize,
showspaces=false,
showstringspaces=false,
captionpos=b,
breaklines=true
}


%\lstdefinestyle{CStyle}{
%    backgroundcolor=\color{backgroundColour},   
%    commentstyle=\color{mGreen},
%    keywordstyle=\color{magenta},
%    numberstyle=\tiny\color{mGray},
%    stringstyle=\color{mPurple},
%    basicstyle=\footnotesize,
%    breakatwhitespace=false,         
%    breaklines=true,                 
%    captionpos=b,                    
%    keepspaces=true,                 
%    numbers=left,                    
%    numbersep=5pt,                  
%    showspaces=false,                
%    showstringspaces=false,
%    showtabs=false,                  
%    tabsize=2,
%    language=C
%}


\title{		\textbf{Trabajo Práctico 1}\\
			\textbf{Conjunto de instrucciones MIPS}
			}

\author{	Lucas Medrano, \textit{Padrón Nro. 99247}                     	\\
            \texttt{ lucasmedrano97@gmail.com }                           		\\
            Federico Álvarez, \textit{Padrón Nro. 99266}                 	\\
            \texttt{ fede.alvarez1997@gmail.com }                                 	\\[2.5ex]
            \normalsize{Grupo Nro. \quad - 2do. Cuatrimestre de 2018}      	\\
            \normalsize{66.20 Organización de Computadoras}               	\\
            \normalsize{Facultad de Ingeniería, Universidad de Buenos Aires}\\
       }
\date{}

\begin{document}
	\lstset{inputencoding=utf8/latin1} % Incorpora acentos en los listings
	\maketitle
	\thispagestyle{empty}
	\begin{abstract}
		En este trabajo se quiere desarrollar un programa escrito en lenguaje C que implementa un algoritmo de Quicksort. Dicho programa ordena alfabéticamente o numéricamente las líneas de un archivo \texttt{.txt}. Se visualizará en pantalla tanto el resultado como los errores que se produzcan. El algoritmo de Quicksort tendrá una implementación en assembler MIPS32, además de la versión en C, para la cual se empleará la convención de pasaje de parámetros establecida en la ABI explicada en clase.
	\end{abstract}
	
	\pagebreak
	\thispagestyle{empty}
	\tableofcontents
	\newpage
	
	\setcounter{page}{1}
	
	\section{Enunciado}
	\begin{figure}[H]
		\centering
		\includegraphics[scale=1, page = 1, clip, trim=20mm 36mm 20mm 35mm]{files/enunciado.pdf}
	\end{figure}
	
	\newpage
	\begin{figure}[H]
		\centering
		\includegraphics[scale=1, page = 2, clip, trim=20mm 36mm 20mm 20mm]{files/enunciado.pdf}
	\end{figure}
	
	\newpage
	\begin{figure}[H]
		\centering
		\includegraphics[scale=1, page = 3, clip, trim=20mm 36mm 20mm 20mm]{files/enunciado.pdf}
	\end{figure}
	
	\newpage
	\begin{figure}[H]
		\centering
		\includegraphics[scale=1, page = 4, clip, trim=20mm 36mm 20mm 20mm]{files/enunciado.pdf}
	\end{figure}
	
	\newpage
	\begin{figure}[H]
		\centering
		\includegraphics[scale=1, page = 5, clip, trim=20mm 36mm 20mm 20mm]{files/enunciado.pdf}
	\end{figure}
	
	\section{Desarrollo}
	
		El programa puede tomar opciones de entrada para indicar el tipo de ordenamiento (alfabético o numérico) y argumentos que designan la salida y el archivo a ordenar.
		
	\subsection{Implementación}
		
		Los errores se definen como cadenas de caracteres constantes. Los mensajes de error se llaman con una función a la que se le pasa el mensaje a mostrar y el número que le corresponde al error.
		
		Las invocaciones a la línea de comandos (como el pedido de versión o de ayuda) también se guardan en cadenas. El tipo de mensaje a mostrar es pasado a la función que los visualiza.
		
		El Quicksort diferencia el tipo de ordenamiento que se hará en un método de ordenamiento general. Esto se logra con un entero identificador: si vale 1, el ordenamiento es numérico; para cualquier otro valor se realizará un ordenamiento alfabético. En el primer caso se utilizará una función atoi para pasar las cadenas de caracteres a números enteros. En el otro caso se compararán las cadenas entre sí.
		
		Al inicio de la función \texttt{main} se verifican los argumentos ingresados para ver si coinciden con las opciones requeridas.	Así se puede responder de acuerdo con el comportamiento elegido.
		
		Se elegió cambiar el nombre de la función qsort porque esta está reservada por C. Se llamó orgaqsort a la función que hicimos en C, y orgaqsortassembly a la que hicimos en assembly. 
		
	\subsection{Pruebas}
	
	\subsubsection{Entradas incorrectas}
	Ingresando sólo el nombre del programa ejecutable:
	\begin{verbatim}
$ qsort
qsort: La cantidad de parametros no es la correcta.
Intente 'qsort -h' para mas informacion
	\end{verbatim}
	
	Ingresando un argumento inválido:
	\begin{verbatim}
$ qsort -w
qsort: La combinacion de parametros no es la correcta.
Intente 'qsort -h' para mas informacion
	\end{verbatim}
	
	Ingresando más argumentos de los necesarios:
	\begin{verbatim}
$ qsort f f f f f
qsort: La cantidad de parametros no es la correcta.
Intente 'qsort -h' para mas informacion
	\end{verbatim}
	
	Ingresando un archivo que no existe:
	\begin{verbatim}
$ qsort -o - numbers.txt
El archivo que quiere ordenar no existe
	\end{verbatim}
	
	Ingresando el orden incorrecto de parámetros (primero \texttt{-o} y luego \texttt{-n}):
	\begin{verbatim}
$ qsort -o -n - numeros.txt
qsort: La combinacion de parametros no es la correcta.
Intente 'qsort -h' para mas informacion
	\end{verbatim}
	Aquí detecta que la opción de ordenamiento alfabético tiene argumentos de más.
	
	
	
	\subsubsection{Ordenando líneas}
	Ingresando un archivo para ordenar alfabéticamente:
	\begin{verbatim}
$ qsort -o - zeta.txt
zzzzzzzzzzzz a
zzzzzzzzzzzz b
zzzzzzzzzzzz sabia que Asuntos Internos le tendia una trampa
	\end{verbatim}
	
	\begin{verbatim}
$ qsort -o - numeros.txt
1
10
2
3
4
5
6
7
8
9
	\end{verbatim}
	
	Ingresando el nombre del archivo que se desea para salida de un ordenamiento alfabético:
	\begin{verbatim}
$ qsort -o orden_alfabetico.txt numeros.txt
	\end{verbatim}
	Se genera un archivo con el nombre \texttt{orden\_alfabetico.txt} que contiene las palabras de \texttt{numeros.txt} en orden alfabético.
	
	\vspace*{12pt}
	Ingresando un archivo para ordenar numéricamente:
	\begin{verbatim}
$ qsort -n -o - zeta.txt
zzzzzzzzzzzz b
zzzzzzzzzzzz sabia que Asuntos Internos le tendia una trampa
zzzzzzzzzzzz a
	\end{verbatim}
	
	\begin{verbatim}
$ qsort -n -o - numeros.txt
1
2
3
4
5
6
7
8
9
10
	\end{verbatim}
	
	Ingresando el nombre del archivo que se desea para salida de un ordenamiento numérico:
	\begin{verbatim}
$ qsort -n -o orden_numerico.txt numeros.txt
	\end{verbatim}
	Se genera un archivo con el nombre \texttt{orden\_alfabetico.txt} que contiene las palabras de \texttt{numeros.txt} en orden numérico.
	
	\section{Conclusiones}
	
	Uno de los principales problemas de este trabajo, fue lograr entender como funcionaban las herramientas usadas, como el emulador y el lenguaje, algo que es dificil al principio. Además nos acostumbramos a las convenciones usadas para estos casos.
	Otra cosa que nos permitió aprender fue combinar archivos en C y archivos en lenguaje assembly. Cosa que nunca antes habíamos hecho.
	Sacamos del trabajo buenas prácticas, como ir probando las funciones del codigo assembly por separado, ya que una vez que está todo junto, es mas complejo.
	
	Se entregan dos scripts. Uno en C, con una implementacion de qsort en C y otras funciones necesarias; y uno en assembly sólo con la funciones qsort, y comparar.
\end{document}
